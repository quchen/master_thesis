% TikZ
\usepackage{tikz}
\usetikzlibrary{
	arrows,
	backgrounds,
	calc,
	decorations.markings,
	decorations.pathmorphing,
	fit,
	intersections,
	patterns,
	positioning,
	through
}

% Arc around a point
\newcommand\TikzArcAround[3]{
	% #1: radius
	% #2: start angle
	% #3: delta angle
	% Example: \draw (1,0) \ArcAround{2}{60}{120}
	++(#2:#1)
	arc[radius=#1, start angle=#2, delta angle=#3]
}

% Helper coordinate system
\newcommand\TikzCoSy[4]{
	% Arguments: integers: x1, x2, y1, y2 ("x range, y range")
	% Lower numbers first!
	(#1,#3) grid (#2,#4)
}
\newcommand\TikzCoSyLabels[4] {
	\foreach \x in {#1,...,#2} {
		node at (\x,0) {\x}
	}
	\foreach \y in {#3,...,#4} {
		\ifnum \y = 0
		\else
			node at (0,\y) {\y}
		\fi
	}
}
\newcommand\TikzFullCoSy[4] {
	\TikzCoSy{#1}{#2}{#3}{#4}
	\TikzCoSyLabels{#1}{#2}{#3}{#4}
}

% Draws a cross at the current location with radius #1.
\newcommand\TikzCross[1]{
	% 0.707107/2 = 1/sqrt(2) for the rotation to get the "x" shape
	+( #1/0.707107/2, #1/0.707107/2) -- +(-#1/0.707107/2,-#1/0.707107/2)
	+( #1/0.707107/2,-#1/0.707107/2) -- +(-#1/0.707107/2, #1/0.707107/2)
	+(0,0)
}

\newcommand\TikzClipMask{
	\draw [draw=black, fill=orange, opacity=0.5, draw opacity=0.8]
}

% Useful for decorations.
% Example:
% mark=between positions 0.0 and 1.0 step 0.1 with {\node (node\pgfkeysvalueof{\SequenceNumber}) [label=below right:{$y_\pgfkeysvalueof{\SequenceNumber}$}] {};}
% Adds nodes named "nodeX" at evenly spaced intervals to a path.
\newcommand\SequenceIndex{\pgfkeysvalueof{/pgf/decoration/mark info/sequence number}}

% Parse mathematical expressions
\newcommand\TikzMath[1] {
	\pgfmathparse{#1}
	\pgfmathresult
}

% Use stealth arrowheads as default
\tikzstyle{every picture}+=[>=stealth]


% Draw arrow at arbitrary position using ->-=0.5 etc
\tikzset{->-/.style={decoration={
	markings,
	mark=at position #1 with {\arrow{>}}},postaction={decorate}}}












