\chapter{Appendix}

\section{Calculation of Fokker-Planck propagator moments}
\label{sec:fp moments}

The Fokker-Planck (FP) propagator has the moments
\begin{align}
	\label{eqn:fp-moment-integral}
	M_{ij;c} =
		\inflint\d (\d q)
		\inflint\d (\d p)
		G_c(q+\d q,p+\d p|q,p; \d t)\d q^i\d p^j ~.
\end{align}
%
The following describes how to calculate these up to relevant orders. For the system discussed in this thesis \RefEqn{eqn:underdamped sde epsilon} \cite{flow-paper}, \(G_c\) takes the form
%
\begin{align}
	\label{eqn:fp-propagator}
	&G_c(q,p,\d q,\d p,\d t) = \\ \notag
	& \frac1{2 \pi  \d t \varepsilon  \Gamma (c \d p +p)}
	  ~\exp \Bigl(\frac1{2 \d t} \\ \notag
	%
	&\quad \times \Bigl(
		-\frac{(\d q-\d t (c \d p +p))^2}{\varepsilon ^2} \\ \notag
		&\qquad +\frac{\left(\d t (c \d p +p) \gamma (c \d p +p)+2 c \d t  \Gamma (c \d p +p) \Gamma '(c \d p +p)+\d p+\d t v'(c \d q+q)\right)^2}{\Gamma (c \d p +p)^2} \\ \notag
		&\qquad +2 \d t^2 c  \left(-(c \d p +p) \gamma '(c \d p +p)-\gamma (c \d p +p)\right) \\ \notag
		&\qquad -2 \d t^2 c ^2 \left(\Gamma (c \d p +p) \Gamma ''(c \d p +p)+\Gamma '(c \d p +p)^2\right)\Bigr)\Bigr)
\end{align}


\subsection{i-th dq moment}

Calculation of the \(i\)-th \(\d q\) moment is a straightforward application of the method of steepest descent \TODO{Ref}
\begin{align}
	\inflint\d q f(q)e^{\frac1\varepsilon g(q)}
	= \lim_{\varepsilon\to0} f(q_{\text{max}}) e^{\varepsilon g(q_{\text{max}})}\sqrt{\frac{2\pi}{-\varepsilon g''(q_{\text{max}})}}
\end{align}
where \(q_{\text{max}}\) denotes the extremum of \(g(q)\).


Here the ``small'' quantity is \(\varepsilon\), which appears only in the prefactor and the first summand in the exponential in \RefEqn{eqn:fp-propagator}; \(q_{\text{max}}\) can be seen to be \(\d t (c\,\d p+p)\) from the propagator's exponential term. Steepest descent then, using \(\tilde p_c = c\,\d p + p\) as a shorthand, yields
\begin{align}
	\label{eqn:dq-integration-finished}
	& \frac{\d t^{i-\frac12} \tilde p_c^i}{\sqrt{2 \pi } \Gamma (\tilde p_c)} \exp \Bigl(
		\\ &\qquad \notag
		\phantom{+}\d t c  \left(\tilde p_c \gamma '(\tilde p_c)
		+\gamma (\tilde p_c)\right)
		+\d t c ^2 \left(\Gamma (\tilde p_c) \Gamma ''(\tilde p_c)
		+\Gamma '(\tilde p_c)^2\right)
		\\ &\qquad \notag
		-\frac{\left(\d t \tilde p_c \gamma (\tilde p_c)
		+2 \d t c  \Gamma (\tilde p_c) \Gamma '(\tilde p_c)
		+\d t v'(c  \tilde p_c \d t +q)+\d p\right)^2}{2 \d t \Gamma (\tilde p_c)^2}
	 \Bigr)
\end{align}
as the intermediate result for \RefEqn{eqn:fp-moment-integral} after the \(\d(\d q)\) integration. (The \(\varepsilon\) fall away naturally without even taking the limit \(\varepsilon\to0\).)



\subsection{j-th dp moment}

For the \(\d p\) moments the approach is similar, but a little more complicated:
%
\begin{enumerate}
	\item The standard formula for steepest descent assumes that the exponential's prefactor is independent of the ``small'' quantity (\(\d t\) in this case). This is not the case for the previously obtained result after \(\d(\d q)\) integration.
	\item Terms quickly grow with increasing order. While it must be made sure that no significant terms are omitted, calculating ``too much'' and dropping unnecessary terms afterwards introduces a lot of work and potential oversights. For that reason one should very carefully keep track of the orders of the terms to be calculated; the criteria for this turn out to be quite different in each order.
\end{enumerate}
%
The expression of interest is
\begin{align}
	\label{eqn:dp-moments-integral}
	I_k(\d t) = \inflint \d (\d p) \, \d p^k f_{\d t}(\d p) \, e^{\frac1{\d t}g(\d p)}
\end{align}
where the integrand ``\(fe^{g/\d t}\)'' is given by \RefEqn{eqn:dq-integration-finished}, which as you recall is the already integrated part of \RefEqn{eqn:fp-moment-integral}; note the potential \(\d t\) dependency of \(f\) denoted by an index. By substitution \(\d p\rightarrow \d p'\sqrt{\d t}\) and following that Taylor expansion of \(f\) and \(g\) around \(\d p'\sqrt{\d t} = \underline{\d p'}\sqrt{\d t}\), this can be transformed into
\begin{align}
	\label{eqn:I_k(dt)}
	I_k(\d t) &=
		\sqrt{\d t}^{k+1}
		\exp\left(\frac1{\d t}g\left(\underline{\d p'}\sqrt{\d t}\right)\right)
		\inflint \d (\d p') \d p'^k
		\\ &\qquad \notag \times
		\exp\left(\frac{\d p'^2}2 g''(\underline{\d p'}\sqrt{\d t})\right)
		\exp\left(\frac{\d p'^3\sqrt{\d t}}2 g'''(\underline{\d p'}\sqrt{\d t})\right)
		\left(1 + \Cal O(\d p'^4 \d t^1)\right)
		\\ &\qquad \notag \times
		\left(
			f_{\d t}(\underline{\d p'}\sqrt{\d t})
			+ f_{\d t}'(\underline{\d p'}\sqrt{\d t})\d p'\sqrt{\d t}
			+ f_{\d t}''(\underline{\d p'}\sqrt{\d t})\frac{\d p^2\d t}2
			+ \Cal O(\d p'^3\d t^{3/2})
		\right)
\end{align}
This intermediate result resembles a corresponding expression in the method of steepest descent: suppose \(f\) does not have an implicit \(\d t\) dependence (in the present notation: does not depend on the index), then the integral would simply be Gaussian in the limit \(\d t\to0\), and we would recover the standard formula.

The following integral will prove handy later (\(a < 0\)):
\begin{align}
	\label{eqn:gaussian_moments}
	J_n = \inflint \d x \, x^n e^{\frac{a x^2}2}
	=
	\Cases{
		\sqrt{\frac{2^{n-1}}{-a^{n+1}}} \Gamma\left(\frac{n+1}2\right)
			& n \text{ even}
		\\
		0 & n \text{ odd}
	}
\end{align}
(derivation: substitute the exponent by a linear expression, leading to the usual \(\Gamma\) integral representation.) The first couple of values are
\begin{align}
	J_0 = \sqrt{\frac{2\pi}{-a}}
	\qquad
	J_2 = \sqrt{\frac{2\pi}{-a^3}}
	\qquad
	J_4 = \sqrt{\frac{18\pi}{-a^5}} ~.
\end{align}

\TODO{Mention that $p_{max} = 0$ now}

\subsubsection{Zeroth dp moment}

The zeroth moment appears with an already given factor of \(\d t\) in \TODO{ref}, therefore the result here only has to be considered in \(\Cal O(\d t^0)\) to get an overall \(\Cal O(\d t^1)\) expression.
%
\begin{align*}
	I_0(\d t)
	&=
	\sqrt{\d t}\exp\left(\frac1{\d t}g(0)\right)
	\inflint\d(\d p')\exp\left(\frac{\d p'^2}2g''(0)\right)
	\left(1+\Cal O(\d p'^3\d t^{1/2})\right)
	\\&\qquad
	\times\Bigl(
	f(0)
	+ \cancel{f'(0)\d p'\sqrt{\d t}}
	+ f''(0)\frac{\d p'^2\d t}2
	+ \Cal O(\d p'^3\d t^{3/2})
	\Bigr)
	\\
	&= f(0)e^{\frac{g(0)}{\d t}}\sqrt{\frac{2\pi\d t}{-g''(0)}} + \Cal O(\d t^1)
\end{align*}
%
Inserting the expressions for \(f\) and \(g\) (obtained by comparing \RefEqn{eqn:dp-moments-integral} with \RefEqn{eqn:dq-integration-finished}) and using \RefEqn{eqn:gaussian_moments} to calculate/cancel out the integrals, then gives
%
\begin{align}
	M_{i0,c} = \d t^i p^i + \Cal O(\d t^{i+1})
\end{align}

\subsubsection{First dp moment}

Unlike the zeroth order, the \(\d p^1\) moment does not have a \(\d t\) prefactor in the entropy formula; expansion therefore has to be up to \(\Cal O(\d t^1)\) here. The surprising property of this moment is that the Gaussian exponential is \emph{not} the only exponential contributing to the result up to the desired order: the term \(\exp(\propto\d p'^3)\) needs to be considered as well, otherwise the result additionally contains an incorrect \(c\)-dependent expression.
%
\begin{align*}
	I_1(\d t)
	&=
	\d t\exp\left(\frac1{\d t}g(0)\right)
	\inflint\d(\d p')\d p'
	\\&\qquad\times
		\exp\left(\frac{\d p'^2}2g''(0)\right)
		\exp\left(\frac{\d p'^3\sqrt{\d t}}2g'''(0)\right)
		\left(1+\Cal O(\d p'^4\d t^1)\right)
	\\&\qquad
	\times\Bigl(
	  f(0)
	+ f'(0)\d p'\sqrt{\d t}
	+ \Cal O(\d p'^2\d t^1)
	\Bigr)
	\\
	&= f'(0)e^{\frac{g(0)}{\d t}}\sqrt{\frac{2\pi\d t^3}{-g''(0)^3}}
	   + \frac{f(0)g'''(0)}6e^{\frac{g(0)}{\d t}}\sqrt{\frac{3^22\pi\d t^3}{-g''(0)^5}}
	   + \Cal O(\d t^2)
\end{align*}
%
Note how in the second root the exponents of \(\d t\) and \(g''\) do not match. The reason for this is that there is an additional \(1/\d t\) hidden in \(f(0\) that appears only in the second summand, canceling the \(\d t^5\) in the root to \(\d t^3\); as a consequence, both summands contribute to the end result to the same order.

Again, \RefEqn{eqn:gaussian_moments} allows us to calculate the integrals, leading to
%
\begin{align}
	M_{i1,c} = -\d t^{i+1}\left(
		p^{i+1}\gamma(p)
		+ p^iV'(q)
		- p^{i-1}ci\Gamma(p)^2\right)
	+ \Cal O(\d t^2)
\end{align}
%
For \(i = 0\), the only case relevant here because it contains a term of order \(\d t^1\), the odd-looking term involving \(ci\) vanishes.

\subsubsection{Second dp moment}

\begin{align*}
	I_2(\d t)
	&=
	\d t^{3/2}\exp\left(\frac1{\d t}g(0)\right)
	\inflint\d(\d p')\exp\left(\frac{\d p'^2}2g''(0)\right)
	\left(1+\Cal O(\d p'^3\d t^{1/2})\right)
	\\&\qquad
	\times\Bigl(
	  f(0)
	+ \cancel{f'(0)\d p'\sqrt{\d t}}
	+ f''(0)\frac{\d p'^2\d t}2
	+ \Cal O(\d p'^3\d t^{3/2})
	\Bigr)
	\\
	&=   f(0)e^{\frac{g(0)}{\d t}}\sqrt{\frac{2\pi\d t^3}{-g''(0)^3}}
	   + \Cal O(\d t^2)
\end{align*}
%
and one last time \RefEqn{eqn:gaussian_moments} is used for integration,
%
\begin{align}
	M_{i2,c} = \d t^{i+1} p^i \Gamma(p)^2 + \Cal O(\d t^2)
\end{align}

\subsubsection{Higher-order dp moments}

Since terms of higher than \(\d t^1\) aren't interesting for short-time entropy production, and keeping in mind that \(f_{\d t}\) involves a factor of \(\d t^{-1/2}\), it can be seen that for \(k>2\) \RefEqn{eqn:I_k(dt)} will always yield a result of negligible order. Therefore, the \(\d p\) moments of power greater than~\(2\) do not appear in short-time entropy and need not be considered.

\subsubsection{Summary of the moments}

Taking the previous sections together, it can be seen that the moments are as follows:
%
\begin{align}
	M_{00;c} &= 1 + \Cal O(\d t) \\
	M_{01;c} &= (-p\gamma(p) - V'(q)) \d t + \Cal O(\d t^2) \\
	M_{02;c} &= \Gamma(p)^2 \d t + \Cal O(\d t^2) \\
	M_{ij;c} &= \Cal O(\d t^{i+1}) \quad (i,j > 0)
\end{align}


\section{Abbreviations}


\begin{tabular}{r l}
	DE & Differential equation \\
	FP & Fokker-Planck \\
	SDE & Stochastic Differential equation \\
	SF & Spinney and Ford, authors of \cite{sf}
\end{tabular}


