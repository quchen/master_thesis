\chapter{Appendix}

\section{Calculation of Fokker-Planck propagator moments}

The following describes how to calculate the \((i,j)\)-th \((\d q, \d p)\) moment of the Fokker-Planck (FP) propagator. For the system discussed in this thesis \TODO{ref to eqn of motion} \TODO{ref to paper}, it has the rather long form

\begin{align*}
	&G_a(q,p,\d q,\d p,\d t) = \\
	& \frac1{2 \pi  \d t \varepsilon  \Gamma (a \d p +p)}
	  ~\exp \Bigl(\frac1{2 \d t} \\
	%
	&\quad \Bigl(
		-\frac{(\d q-\d t (a \d p +p))^2}{\varepsilon ^2} \\
		&\qquad +\frac{\left(\d t (a \d p +p) \gamma (a \d p +p)+2 a \d t  \Gamma (a \d p +p) \Gamma '(a \d p +p)+\d p+\d t v'(a \d q+q)\right)^2}{\Gamma (a \d p +p)^2} \\
		&\qquad +2 \d t^2 a  \left(-(a \d p +p) \gamma '(a \d p +p)-\gamma (a \d p +p)\right) \\
		&\qquad -2 \d t^2 a ^2 \left(\Gamma (a \d p +p) \Gamma ''(a \d p +p)+\Gamma '(a \d p +p)^2\right)\Bigr)\Bigr)
\end{align*}

In the following, a modified version of the method of steepest descent will be derived, and then applied to the propagator to calculate the moments up to the needed order.


\subsection{Generalized method of steepest descent}

The moments of the FP propagator can be calculated similar to how the method of steepest descent is derived. However, there are a few details making it a little more difficult:

\begin{enumerate}
	\item The standard formula for steepest descent assumes that the exponential's prefactor is independent of the ``small'' quantity. This is not the case for the FP propagator, so a new formula has to be developed, albeit in the same spirit.
	\item Terms quickly grow with increasing order. While it must be made sure that no significant terms are omitted, calculating ``too much'' and dropping unnecessary terms afterwards introduces a lot of work and potential oversights. For that reason one should very carefully keep track of the orders of the terms to be calculated; the criteria for this turn out to be quite different in each order.
\end{enumerate}


\subsection{i-th dq moment}

Calculation of the \(i\)-th \(\d q\) moment is a straightforward application of the method of steepest descent

\subsection{j-th dp moment}