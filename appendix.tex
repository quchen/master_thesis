\chapter{Appendix}

\renewcommand{\thechapter}{\Alph{chapter}.}
\renewcommand{\thesection}{\Alph{chapter}.\arabic{section}}
\renewcommand{\thesubsection}{\Alph{chapter}.\arabic{section}.\arabic{subsection}}
\renewcommand{\theequation}{\Alph{chapter}.\arabic{equation}}

\section{General expressions for environmental entropy production}
\label{sec:general SEnv expressions}

The general expressions obtained for \SF{} and \HF{} are different; despite this fact, they lead to identical results in the investigated scenarios. Here are the full expressions, with the ambiguities \({\color{red}a}\) and \({\color{red}b}\) left as free and independent parameters and differing terms boxed, for \(\d\SEnv\):
\\
\textbf{Spinney-Ford:}
\begin{equation*}\begin{split}
	\d\SEnv^\SF &= 
	{\color{red}a} \Bigl(-\frac{\d p^2 p \mu '(p)}{\Gamma (p)^2}+\frac{2 \d p^2 p \mu (p) \Gamma '(p)}{\Gamma (p)^3}-\frac{\d p^2 \mu (p)}{\Gamma (p)^2}+\frac{2 \d p^2 \Gamma '(p) V'(q)}{\Gamma (p)^3}-\frac{3 \d p \Gamma '(p)}{\Gamma (p)}
		\\&\qquad\quad
		+ \d t p \mu '(p)-\frac{2 \d t p \mu (p) \Gamma '(p)}{\Gamma (p)}+\d t \mu (p)-\frac{2 \d t \Gamma '(p) V'(q)}{\Gamma (p)}\Bigr)
	\\&\quad
	+{\color{red}b} \Bigl(\frac{\d p^2 p \mu '(p)}{\Gamma (p)^2}-\frac{2 \d p^2 p \mu (p) \Gamma '(p)}{\Gamma (p)^3}+\frac{\d p^2 \mu (p)}{\Gamma (p)^2}-\frac{3 \d p^2 \Gamma ''(p)}{\Gamma (p)} +\frac{3 \d p^2 \Gamma '(p)^2}{\Gamma (p)^2}
		\\&\qquad\quad
		\boxed{+\frac{2 \d p^2 \Gamma '(p) V'(q)}{\Gamma (p)^3}}-\frac{3 \d p \Gamma '(p)}{\Gamma (p)}-\d t p \mu '(p)+\frac{2 \d t p \mu (p) \Gamma '(p)}{\Gamma (p)}
		\\&\qquad\quad
		-\d t \mu (p)\boxed{-\frac{2 \d t \Gamma '(p) V'(q)}{\Gamma (p)}}\Bigr)
	\\&\quad
	+({\color{red}a}^2 - {\color{red}b}^2) \left(-\frac{5 \d p^2 \Gamma ''(p)}{2 \Gamma (p)}+\frac{5 \d p^2 \Gamma '(p)^2}{2 \Gamma (p)^2}+\d t \Gamma (p) \Gamma ''(p)-\d t \Gamma '(p)^2\right)
	\\&\quad
	-\frac{\d p^2 p \mu '(p)}{\Gamma (p)^2}+\frac{2 \d p^2 p \mu (p) \Gamma '(p)}{\Gamma (p)^3}-\frac{\d p^2 \mu (p)}{\Gamma (p)^2}+\frac{\d p^2 \Gamma ''(p)}{2 \Gamma (p)}
	\\&\quad
	-\frac{\d p^2 \Gamma '(p)^2}{2 \Gamma (p)^2}\boxed{-\frac{2 \d p^2 \Gamma '(p) V'(q)}{\Gamma (p)^3}}-\frac{2 \d p p \mu (p)}{\Gamma (p)^2}+\frac{\d p \Gamma '(p)}{\Gamma (p)}-\frac{2 \d t p \mu (p) V'(q)}{\Gamma (p)^2}
\end{split}\end{equation*}
%
\\
\textbf{Flow model:}
\begin{equation*}\begin{split}
	\d\SEnv^\HF &=
	{\color{red}a} \Bigl(-\frac{\d p^2 p \mu '(p)}{\Gamma (p)^2}+\frac{2 \d p^2 p \mu (p) \Gamma '(p)}{\Gamma (p)^3}-\frac{\d p^2 \mu (p)}{\Gamma (p)^2}+\frac{2 \d p^2 \Gamma '(p) V'(q)}{\Gamma (p)^3}-\frac{3 \d p \Gamma '(p)}{\Gamma (p)}
		\\&\quad\qquad
		+\d t p \mu '(p)-\frac{2 \d t p \mu (p) \Gamma '(p)}{\Gamma (p)}+\d t \mu (p)-\frac{2 \d t \Gamma '(p) V'(q)}{\Gamma (p)}\Bigr)
	\\&\quad
	+{\color{red}b} \Bigl(\frac{\d p^2 p \mu '(p)}{\Gamma (p)^2}-\frac{2 \d p^2 p \mu (p) \Gamma '(p)}{\Gamma (p)^3}+\frac{\d p^2 \mu (p)}{\Gamma (p)^2}-\frac{3 \d p^2 \Gamma ''(p)}{\Gamma (p)}+\frac{3 \d p^2 \Gamma '(p)^2}{\Gamma (p)^2}
		\\&\quad\qquad
		\boxed{+\frac{4 \d p^2 \Gamma '(p) V'(q)}{\Gamma (p)^3}}-\frac{3 \d p \Gamma '(p)}{\Gamma (p)}-\d t p \mu '(p)+\frac{2 \d t p \mu (p) \Gamma '(p)}{\Gamma (p)}
		\\&\quad\qquad
		-\d t \mu (p)\boxed{-\frac{4 \d t \Gamma '(p) V'(q)}{\Gamma (p)}}\Bigr)
	\\&\quad
	+({\color{red}a}^2-{\color{red}b}^2) \left(-\frac{5 \d p^2 \Gamma ''(p)}{2 \Gamma (p)}+\frac{5 \d p^2 \Gamma '(p)^2}{2 \Gamma (p)^2}+\d t \Gamma (p) \Gamma ''(p)-\d t \Gamma '(p)^2\right)
	\\&\quad
	-\frac{\d p^2 p \mu '(p)}{\Gamma (p)^2}+\frac{2 \d p^2 p \mu (p) \Gamma '(p)}{\Gamma (p)^3}-\frac{\d p^2 \mu (p)}{\Gamma (p)^2}+\frac{\d p^2 \Gamma ''(p)}{2 \Gamma (p)}
	\\&\quad
	-\frac{\d p^2 \Gamma '(p)^2}{2 \Gamma (p)^2}\boxed{-\frac{3 \d p^2 \Gamma '(p) V'(q)}{\Gamma (p)^3}}-\frac{2 \d p p \mu (p)}{\Gamma (p)^2}+\frac{\d p \Gamma '(p)}{\Gamma (p)}-\frac{2 \d t p \mu (p) V'(q)}{\Gamma (p)^2}
	\\&\quad
	\boxed{+\frac{\d t \Gamma '(p) V'(q)}{\Gamma (p)}}
\end{split}\end{equation*}

\section{Calculation of Fokker-Planck propagator moments}
\label{sec:fp moments}

The Fokker-Planck (FP) propagator has the moments
\begin{align}
	\label{eqn:fp-moment-integral}
	M_{ij;c} =
		\inflint\d (\d q)
		\inflint\d (\d p)
		G_c(q+\d q,p+\d p|q,p; \d t)\d q^i\d p^j ~.
\end{align}
%
The following describes how to calculate these up to relevant orders. For the system discussed in this thesis \RefEqn{eqn:underdamped sde epsilon} \cite{flow-paper}, \(G_c\) takes the form
%
\begin{align}
	\label{eqn:fp-propagator}
	&G_c(q,p,\d q,\d p,\d t) = \\ \notag
	& \frac1{2 \pi  \d t \varepsilon  \Gamma (c \d p +p)}
	  ~\exp \Bigl(\frac1{2 \d t} \\ \notag
	%
	&\quad \times \Bigl(
		-\frac{(\d q-\d t (c \d p +p))^2}{\varepsilon ^2} \\ \notag
		&\qquad +\frac{\left(\d t (c \d p +p) \gamma (c \d p +p)+2 c \d t  \Gamma (c \d p +p) \Gamma '(c \d p +p)+\d p+\d t v'(c \d q+q)\right)^2}{\Gamma (c \d p +p)^2} \\ \notag
		&\qquad +2 \d t^2 c  \left(-(c \d p +p) \gamma '(c \d p +p)-\gamma (c \d p +p)\right) \\ \notag
		&\qquad -2 \d t^2 c ^2 \left(\Gamma (c \d p +p) \Gamma ''(c \d p +p)+\Gamma '(c \d p +p)^2\right)\Bigr)\Bigr)
\end{align}


\subsection{i-th dq moment}

Calculation of the \(i\)-th \(\d q\) moment is a straightforward application of the standard method of steepest descent
\begin{align}
	\label{eqn:steepest-descent}
	\lim_{\varepsilon\to0}\inflint\d q f(q)e^{\frac1\varepsilon g(q)}
	= \lim_{\varepsilon\to0} f(q_{\text{max}}) e^{\varepsilon g(q_{\text{max}})}\sqrt{\frac{2\pi}{-\varepsilon g''(q_{\text{max}})}}
\end{align}
where \(q_{\text{max}}\) denotes the extremum of \(g(q)\).


Here the ``small'' quantity is \(\varepsilon\), which appears only in the prefactor and the first summand in the exponential in \RefEqn{eqn:fp-propagator}; \(q_{\text{max}}\) can be seen to be \(\d t (c\,\d p+p)\) from the propagator's exponential term. Steepest descent then, using \(\tilde p_c = c\,\d p + p\) as a shorthand, yields
\begin{align}
	\label{eqn:dq-integration-finished}
	& \frac{\d t^{i-\frac12} \tilde p_c^i}{\sqrt{2 \pi } \Gamma (\tilde p_c)} \exp \Bigl(
		\\ &\qquad \notag
		\phantom{+}\d t c  \left(\tilde p_c \gamma '(\tilde p_c)
		+\gamma (\tilde p_c)\right)
		+\d t c ^2 \left(\Gamma (\tilde p_c) \Gamma ''(\tilde p_c)
		+\Gamma '(\tilde p_c)^2\right)
		\\ &\qquad \notag
		-\frac{\left(\d t \tilde p_c \gamma (\tilde p_c)
		+2 \d t c  \Gamma (\tilde p_c) \Gamma '(\tilde p_c)
		+\d t v'(c  \tilde p_c \d t +q)+\d p\right)^2}{2 \d t \Gamma (\tilde p_c)^2}
	 \Bigr)
\end{align}
as the intermediate result for \RefEqn{eqn:fp-moment-integral} after the \(\d(\d q)\) integration. (The \(\varepsilon\) fall away naturally without even taking the limit \(\varepsilon\to0\).)



\subsection{j-th dp moment}

For the \(\d p\) moments the approach is similar, but a little more complicated:
%
\begin{enumerate}
	\item The standard formula for steepest descent assumes that the exponential's prefactor is independent of the ``small'' quantity (\(\d t\) in this case). This is not the case for the previously obtained result after \(\d(\d q)\) integration.
	\item Terms quickly grow with increasing order. While it must be made sure that no significant terms are omitted, calculating ``too much'' and dropping unnecessary terms afterwards introduces a lot of work and potential oversights. For that reason one should very carefully keep track of the orders of the terms to be calculated; the criteria for this turn out to be quite different in each order.
\end{enumerate}
%
The expression of interest is
\begin{align}
	\label{eqn:dp-moments-integral}
	I_k(\d t) = \inflint \d (\d p) \, \d p^k f_{\d t}(\d p) \, e^{\frac1{\d t}g(\d p)}
\end{align}
where the integrand ``\(fe^{g/\d t}\)'' is given by \RefEqn{eqn:dq-integration-finished}, which as you recall is the already integrated part of \RefEqn{eqn:fp-moment-integral}; note the potential \(\d t\) dependency of \(f\) denoted by an index. By substitution \(\d p\rightarrow \d p'\sqrt{\d t}\) and following that Taylor expansion of \(f\) and \(g\) around \(\d p'\sqrt{\d t} = \underline{\d p'}\sqrt{\d t}\), this can be transformed into
\begin{align}
	\label{eqn:I_k(dt)}
	I_k(\d t) &=
		\sqrt{\d t}^{k+1}
		\exp\left(\frac1{\d t}g\left(\underline{\d p'}\sqrt{\d t}\right)\right)
		\inflint \d (\d p') \d p'^k
		\\ &\qquad \notag \times
		\exp\left(\frac{\d p'^2}2 g''(\underline{\d p'}\sqrt{\d t})\right)
		\exp\left(\frac{\d p'^3\sqrt{\d t}}2 g'''(\underline{\d p'}\sqrt{\d t})\right)
		\left(1 + \Cal O(\d p'^4 \d t^1)\right)
		\\ &\qquad \notag \times
		\left(
			f_{\d t}(\underline{\d p'}\sqrt{\d t})
			+ f_{\d t}'(\underline{\d p'}\sqrt{\d t})\d p'\sqrt{\d t}
			+ f_{\d t}''(\underline{\d p'}\sqrt{\d t})\frac{\d p^2\d t}2
			+ \Cal O(\d p'^3\d t^{3/2})
		\right)
\end{align}
This intermediate result resembles a corresponding expression in the method of steepest descent: suppose \(f\) does not have an implicit \(\d t\) dependence (in the present notation: does not depend on the index), then the integral would simply be Gaussian in the limit \(\d t\to0\), standard formula \RefEqn{eqn:steepest-descent} would be recovered. Compared to the \(\d q\) case, \(\d p_\text{max}\) could not be easier, as it turns out to be \(0\).

The following integral will prove handy later (\(a < 0\)):
\begin{align}
	\label{eqn:gaussian_moments}
	J_n = \inflint \d x \, x^n e^{\frac{a x^2}2}
	=
	\Cases{
		\sqrt{\frac{2^{n-1}}{-a^{n+1}}} \Gamma\left(\frac{n+1}2\right)
			& n \text{ even}
		\\
		0 & n \text{ odd}
	}
\end{align}
(derivation: substitute the exponent by a linear expression, leading to the usual \(\Gamma\) integral representation.) The first couple of values are
\begin{align}
	J_0 = \sqrt{\frac{2\pi}{-a}}
	\qquad
	J_2 = \sqrt{\frac{2\pi}{-a^3}}
	\qquad
	J_4 = \sqrt{\frac{18\pi}{-a^5}} ~.
\end{align}





\subsubsection{Zeroth dp moment}

The zeroth moment appears in the context of \(\sigma_{00}\), which is accompanied by \(\d t\) implicitly (due to the shape of the propagator quotients' expansions, namely \RefEqn{eqn:sf entropy production}/\SF{} and \RefEqn{eqn:hf entropy}/\HF{}) in \RefEqn{eqn:dSEnvLoc split up}. Therefore, in order to obtain an overall result of up to \(\Cal O(\d t^1)\), it is sufficient to calculate only \(\Cal O(\d t^0)\) here.
%
\begin{align*}
	I_0(\d t)
	&=
	\sqrt{\d t}\exp\left(\frac1{\d t}g(0)\right)
	\inflint\d(\d p')\exp\left(\frac{\d p'^2}2g''(0)\right)
	\left(1+\Cal O(\d p'^3\d t^{1/2})\right)
	\\&\qquad
	\times\Bigl(
	f(0)
	+ \cancel{f'(0)\d p'\sqrt{\d t}}
	+ f''(0)\frac{\d p'^2\d t}2
	+ \Cal O(\d p'^3\d t^{3/2})
	\Bigr)
	\\
	&= f(0)e^{\frac{g(0)}{\d t}}\sqrt{\frac{2\pi\d t}{-g''(0)}} + \Cal O(\d t^1)
\end{align*}
%
Inserting the expressions for \(f\) and \(g\) (obtained by comparing \RefEqn{eqn:dp-moments-integral} with \RefEqn{eqn:dq-integration-finished}) and using \RefEqn{eqn:gaussian_moments} to calculate/cancel out the integrals, then results in
%
\begin{align}
	M_{i0,c} = \d t^i p^i + \Cal O(\d t^{i+1}) ~ \forall c
\end{align}

\subsubsection{First dp moment}

Unlike the zeroth order, the \(\d p^1\) moment does not have an implicit \(\d t\) prefactor in the entropy formula; expansion therefore has to be up to \(\Cal O(\d t^1)\) here. The surprising property of this moment is that the Gaussian exponential is \emph{not} the only exponential contributing to the result up to the desired order: the term \(\exp(\propto\d p'^3)\) needs to be considered as well, otherwise the result additionally contains an incorrect \(c\)-dependent expression. Using \RefEqn{eqn:gaussian_moments} again allows us to calculate the integrals, leading to
%
\begin{align*}
	I_1(\d t)
	&=
	\d t\exp\left(\frac1{\d t}g(0)\right)
	\inflint\d(\d p')\d p'
	\\&\qquad\times
		\exp\left(\frac{\d p'^2}2g''(0)\right)
		\exp\left(\frac{\d p'^3\sqrt{\d t}}2g'''(0)\right)
		\left(1+\Cal O(\d p'^4\d t^1)\right)
	\\&\qquad
	\times\Bigl(
	  f(0)
	+ f'(0)\d p'\sqrt{\d t}
	+ \Cal O(\d p'^2\d t^1)
	\Bigr)
	\\
	&= f'(0)e^{\frac{g(0)}{\d t}}\sqrt{\frac{2\pi\d t^3}{-g''(0)^3}}
	   + \frac{f(0)g'''(0)}6e^{\frac{g(0)}{\d t}}\sqrt{\frac{3^22\pi\d t^3}{-g''(0)^5}}
	   + \Cal O(\d t^2)
\end{align*}
%
so the moment turns out to be
%
\begin{align}
	M_{i1,c} = -\d t^{i+1}\left(
		p^{i+1}\gamma(p)
		+ p^iV'(q)
		- p^{i-1}ci\Gamma(p)^2\right)
	+ \Cal O(\d t^2) ~.
\end{align}
%
The only case relevant to entropy production is \(i = 0\), as it is the only one containing a term of order \(\d t^1\). This also gets rid of the odd term involving \(c\cdot i\).

\subsubsection{Second dp moment}

The highest moment is, maybe surprisingly, the easiest one of the three required ones: all the complicated terms are of order higher than \(\Cal O(\d t^1)\), and the only significant expression left over is a single square root term:
%
\begin{align*}
	I_2(\d t)
	&=
	\d t^{3/2}\exp\left(\frac1{\d t}g(0)\right)
	\inflint\d(\d p')\exp\left(\frac{\d p'^2}2g''(0)\right)
	\left(1+\Cal O(\d p'^3\d t^{1/2})\right)
	\\&\qquad
	\times\Bigl(
	  f(0)
	+ \cancel{f'(0)\d p'\sqrt{\d t}}
	+ f''(0)\frac{\d p'^2\d t}2
	+ \Cal O(\d p'^3\d t^{3/2})
	\Bigr)
	\\
	&=   f(0)e^{\frac{g(0)}{\d t}}\sqrt{\frac{2\pi\d t^3}{-g''(0)^3}}
	   + \Cal O(\d t^2)
\end{align*}
%
and one last time \RefEqn{eqn:gaussian_moments} is used for integration,
%
\begin{align}
	M_{i2,c} = \d t^{i+1} p^i \Gamma(p)^2 + \Cal O(\d t^2) ~.
\end{align}

\subsubsection{Higher-order dp moments}

Since terms of higher than \(\d t^1\) aren't interesting for short-time entropy production, and keeping in mind that \(f_{\d t}\) involves a factor of \(\d t^{-1/2}\), it can be seen that for \(k>2\) \RefEqn{eqn:I_k(dt)} will always yield a result of negligible order. Therefore, the \(\d p\) moments of power greater than~\(2\) do not appear in short-time entropy and need not be considered, as it is the case for \(\d q\) moments of order greater than \(0\).

\subsubsection{Summary of the moments}

Taking the previous sections together, it can be seen that the moments are as follows:
%
\begin{align*}
	M_{00;c} &= 1 + \Cal O(\d t) \\
	M_{01;c} &= (-p\gamma(p) - V'(q)) \d t + \Cal O(\d t^2) \\
	M_{02;c} &= \Gamma(p)^2 \d t + \Cal O(\d t^2) \\
	M_{ij;c} &= \Cal O(\d t^{i+1}) \quad\text{else}
\end{align*}
%
as claimed in the main text.


\section{Calculation of the global entropy production}
\label{sec:global entropy zero}

To calculate the global entropy production in section~\RefSection{sec:global entropy production}, the integral in question is
%
\begin{equation}
	\langle\d\SEnv\rangle(t)
	= \d t \inflint\d q\inflint\d p\, P(q, p, t) \, \d\SEnvLoc(q, p, \d t)
\end{equation}
%
with
%
\begin{align}
	P(q,p,t) &= \tfrac1Ze^{-\beta\left(\frac{p^2}2-V(q)\right)} \\
	\d\SEnvLoc(q,p,\d t) &= \left(
		- \tfrac12\beta\Gamma(p)^2
		+ \tfrac12\beta^2p^2\Gamma(p)^2
		- \beta p\Gamma(p)\Gamma'(p)
		\right)\,\d t
\end{align}
%
where terms of higher than linear order have been dropped. Noting that
%
\begin{align}
	&\phantom{=} - \tfrac12\beta\Gamma(p)^2
		+ \tfrac12\beta^2p^2\Gamma(p)^2
		- \beta p\Gamma(p)\Gamma'(p) \\
	&= \tfrac12\beta^2p^2\Gamma(p)^2
		- \frac\beta2\left(
		\Gamma^2+p\Gamma(p)\Gamma'(p)
		\right) \\
	&= \underbrace{\tfrac12\beta^2p^2\Gamma(p)^2}_{\equiv A}
		- \underbrace{\frac\beta2\tdq{}p\left(p\Gamma(p)^2\right)}_{\equiv B}
\end{align}
%
the integral can be divided up in two contributions over \(A\) and \(B\); then
%
\begin{equation}\begin{split}
	I_B &= \frac\beta{2Z} \inflint\d p e^{-\beta\left(\frac{p^2}2-V(q)\right)} \tdq{}p\left(p\Gamma(p)^2\right) \\
	&= \frac\beta{2Z}\left(\left[ -\beta p e^{-\beta\left(\frac{p^2}2-V(q)\right)} \right]_{p = -\infty}^\infty - \inflint\d p \; (-\beta p) e^{-\beta\left(\frac{p^2}2-V(q)\right)} p\Gamma(p)^2\right) \\
	&= \frac{\beta^2}{2Z} \inflint\d p e^{-\beta\left(\frac{p^2}2-V(q)\right)} p^2\Gamma^2 \\
	&= I_A
\end{split}\end{equation}
%
so both terms cancel out (before the \(\d q\) integration is even necessary), resulting in
%
\begin{equation}
	\langle\d\SEnv\rangle(t) = 0 ~ \forall t
\end{equation}
%
as claimed in \RefEqn{eqn:global entropy 0}.









\section{Abbreviations}


\begin{tabular}{r l}
	DE & Differential equation \\
	FP & Fokker-Planck \\
	HF & Hamiltonian-Flow-based model for environmental entropy production \\
	SDE & Stochastic Differential equation \\
	SF & Spinney and Ford, authors of \cite{sf}
\end{tabular}






\section*{Thanks}

This thesis would not have been possible without the help of others. I would like to thank Prof.~Dr.~Wolfgang Kinzel for making me part of the group of theoretical physics~3, making the research possible in the first place.

My advisor Prof.~Dr.~Haye Hinrichsen provided the topic along with many important discoveries/ideas (most notably it was him who found the histogram symmetry that lead to the research in chapter~\RefSection{chap:thingie}, and also finding the continuous version of entropy having implausible results -- the motivation behind section~\RefSection{chap:flow}), cross-checking results, constructively judging methods.

Prof.~Dr.~Joel Lebowitz' seminar over my one year stay at Rutgers University was the first research-level contact with the area of statistical physics, and is what made me consider taking this path.

Last but not least I was part of a group of people that will be missed; in the age of internet crawlers they shall be named explicitly (ordered by office location): Dr.~J�rg Reichardt, Steffen Zeeb, Peter Janotta, Dr.~Otti d'Huys, Prof. Dr.~Georg Reents, Marco Winkler, Barbara Schl�gl and Peter Reisenauer.


