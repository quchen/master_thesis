\begin{tikzpicture}
	[>=stealth]

	% coordinate system
	\begin{scope}
		% axes
		\begin{scope}[->]
			\newcommand\Excess{0.2}
			\draw (-\Excess,0) -- (10.5,0); % x
			\draw (0,-\Excess) -- (0,6);  % y
		\end{scope}


		\newcommand{\TickLength}{0.2}
		\newcommand{\LabelOffset}{0.5}
		
		% y axis labels/ticks
		\begin{scope}
			\foreach \y in {1,...,5} {
				\draw (-\TickLength/2, \y) -- (\TickLength/2, \y);
 				\node at (-\LabelOffset, \y) {\(\y\)};
			}
			\node at (0, 6 + \LabelOffset) {\(\Omega\)};
		\end{scope}

		% x axis labels/ticks
		\begin{scope}
			\foreach \i/\x in {0/0, 1/2, 2/3, 3/6, 4/7.5, 5/8, 6/10} {
				\draw (\x, -\TickLength/2) -- (\x, \TickLength/2);
				\node at (\x, -\LabelOffset) {\(\tau_\i\)};
			}
			\node at (0, -2*\LabelOffset) {\(0\)};
			\node at (10, -2*\LabelOffset) {\(T = \tau_{J+1}\)};

			\node at (10.5 + \LabelOffset, 0) {\(t\)};
		\end{scope}

	\end{scope}

	% Path taken
	\begin{scope}
		
		% Actual path
		\begin{scope} [thick]
			\draw (0,1) -- (2,1)
				(2,4) -- (3,4)
				(3,3) -- (6,3)
				(6,5) -- (7.5,5)
				(7.5,1) -- (8,1)
				(8,2) -- (10,2)
				;
			\newcommand{\Radius}{0.03}
			\foreach \xy in {(0,1), (2,1), (2,4), (3,4), (3,3), (6,3), (6,5), (7.5,5), (7.5,1), (8,1), (8,2), (10,2)} {
				\draw[fill=black] \xy circle (\Radius);
			}
		\end{scope}

		% Connecting jump lines
		\begin{scope} [dotted]
			\draw (2,1)-- (2,4)
			      (3,4) -- (3,3) 
				(6,3) -- (6,5) 
				(7.5,5) -- (7.5,1) 
				(8,1) -- (8,2)
				;
		\end{scope}

		% labels
		\begin{scope}
			\newcommand{\LabelOffset}{0.5}
			\node at (1,         1 - 0.5*\LabelOffset) {\small \(c_0 = 1\)};
			\node at (2.5,       4 + 0.5*\LabelOffset) {\small \(c_1 = 4\)};
			\node at (4.5,       3 - 0.5*\LabelOffset) {\small \(c_2 = 3\)};
			\node at (6.75,      5 + 0.5*\LabelOffset) {\small \(c_3 = 5\)};
			\node at (7.75,      1 - 0.5*\LabelOffset) {\small \(c_4 = 1\)};
			\node at (9,         2 + 0.5*\LabelOffset) {\small \(c_5 = 2\)};
		\end{scope}

		% final dashed line at t=T

		\begin{scope}
			\draw [dashed] (10,0) -- (10,6);
		\end{scope}
		
	\end{scope}

\end{tikzpicture}
